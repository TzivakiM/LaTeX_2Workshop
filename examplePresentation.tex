\documentclass[xcolor={dvipsnames}]{beamer}
\usetheme{CambridgeUS}
%if you would like a specific color theme
%\usecolortheme{seahorse}
%if you don't like navigation symbols
%\setbeamertemplate{navigation symbols}{}

%Useful packages
\usepackage[english]{babel}
\usepackage{xcolor,graphicx}
\usepackage{amsmath}
\usepackage{booktabs}


%%%%%%%%%%%%%%%%%%COLOUR GAMES%%%%%%%%%%%%%%%%%%%%%%%%%
%uncomment for default
%Define new colours (UOIT official colours)
\definecolor{uoitbluedark}{HTML}{003580}
\definecolor{uoitbluelight}{HTML}{0082D1}

%changing colours in the main theme
\setbeamercolor{palette primary}{fg=black, bg=gray!60!white}
\setbeamercolor{palette secondary}{fg=black, bg=gray!20!white}
\setbeamercolor{palette tertiary}{fg=white, bg=uoitbluedark}

%changing colours in the titles
\setbeamercolor{frametitle}{fg=uoitbluedark!100!black}
\setbeamercolor{title}{fg=uoitbluedark!100!black}

%changing colours in blocks
\setbeamercolor{block title}{fg=uoitbluedark}
\setbeamercolor{block title example}{fg=uoitbluelight}

%%%%%%%%%%%%%%%%%%%%%%%%%%%%%%%%%%%%%%%%%%%%%%%%%%%%%%%%%%%%%%%%

%%%%%%%%%%%%%%%%TITLES AND THE LIKE%%%%%%%%%%%%%%%%%%%%%%%%%%%

\title[LaTeX Presentations]{Making Presentations in \LaTeX}
\subtitle{A hands-on workshop}
\date{\today}
\author{Margarita Tzivaki}
\logo{\includegraphics[width=.13\textwidth]{logo}}
\institute[ASC]{Academic Skills Club}

%%%%%%%%%%%%%%%%%%%%%%%%%%%%%%%%%%%%%%%%%%%%%%%%%%%%%%%%%%%%%%%%
\begin{document}

\frame{\maketitle}
\section{Text and Organization}
\subsection{Text}
\begin{frame}{Text}{Fun text things}
We can do all kinds of things with text. You can make text \textbf{bold}, 
\emph{italicized}, and \textcolor{blue}{coloured}. In addition it is also 
useful to know how to superscript text A${}^{\text{stuff}}$ or subscript 
B${}_{\text{stuff}}$. 
\end{frame}
\subsection{Organization}
\begin{frame}{Organization}{Look how well it works}
Woot woot!
\end{frame}

\section{Slide essentials}
\subsection{Columns}
\begin{frame}{Columns}

\begin{columns}
\begin{column}{0.5\textwidth}
something
\end{column}
\begin{column}{0.5\textwidth}
something else
\end{column}
\end{columns}

\end{frame}
\subsection{Spacing}
\begin{frame}{Spacing}{Vertical and horizontal}

Some text here with vertical space after

\vspace{2cm}

More text after that and some horizontan space \hspace{1cm} boom!

\end{frame}

\subsection{Lists}
\begin{frame}{Lists and their spacing}{Enumerated lists}
There are two main ways to create lists in LaTeX, enumerate (for numbered lists) 
and itemize.
%As always we have to start the environment 
\begin{enumerate}
%Add items to the list via \item
\item This is a thing
\item This is another thing 
\end{enumerate}

\end{frame}

\begin{frame}{Itemized lists}

The lists don't have to be numbered. Using itemize will give you bullet points 
(but you can change what they are). Here is an example using enumerate. For controlling spacing you will need the command itemsep.

\begin{itemize}
\setlength{\itemsep}{1.5em}
%Again we add items to the list with \item 
\item Stuff
\item More stuff 
\end{itemize}
\end{frame}

\subsection{Blocks}
\begin{frame}{Blocks}{Fun with Lego!}
We can make regular blocks:
\begin{block}
{Blocktitle}{Some important text}
\end{block}

Or we can make alertblocks:
\begin{alertblock}
{Warning!}{Warning message}
\end{alertblock}

Or we can make exampleblocks:
\begin{exampleblock}
{Example Title}{Example text that is very instructive}
\end{exampleblock}
\end{frame}

\subsection{Pretty Pictures}
\begin{frame}

%As always we have to start the environment 
\begin{figure}[h]
%This is how we can center the figure 
\begin{center}
%You can resize the figure using the scale option
%There are options to clip the figure as well 
\includegraphics[scale=1.5]{logo.png}
\end{center}
%Add your caption here 
\caption{We can add captions to our figures as well}
%Close the environment 
\end{figure}

\end{frame}

\subsection{Math and Equations}
\begin{frame}{Using math mode}

As usually one of the primary reasons you will use \LaTeX \ is for writing equations.  

\begin{equation}
F_{net}=ma
\end{equation}

Equations don't have to be numbered. You can disable the numbering by putting 
an asterisk in the begin and end statements 
\begin{equation*}
E=mc^{2}
\end{equation*}
It is also very convenient to write multiple lines in an equation using 
the align environment:
\begin{align*}
2x - 5y &=  8 \\ 
3x + 9y &=  -12
\end{align*}
\end{frame}

\begin{frame}{Using math mode even more}

You can write all sorts of fancy symbols (which can be found on the cheatsheet!)
\begin{equation*}
i\hbar \frac{\partial | \Psi \rangle}{ \partial t}=\hat{H}|\Psi \rangle
\end{equation*}

Math mode can be used inline with text (e.g. $e^{-\lambda x}$) which is very 
convenient. All you need to do is wrap your equation (or whatever you are using)
in dollar signs.

\end{frame}


\subsection{Tables}
\begin{frame}{Making Tables}
Here we will create tables which can be a nice way of presenting data. 

%Begin the tabular environment 
\begin{center}
%The c's indicate center justified text
%Vertical lines will put vertical lines between columns 
\begin{tabular} { c  c  c }
\toprule
Fruit & Quantity & Price \\ \midrule
Apple & 2 & \$2.00 \\ \midrule
Banana & 5 & \$3.50 \\ \midrule
Orange & 8 & \$4.00 \\ 
\bottomrule
\end{tabular}
\end{center}

\end{frame}

\subsection{Footnotes}

\begin{frame}{How to add footnotes}
Here is some text that will need an explanation in a footnote \footnote[frame]{\footnotesize{Here is the associated footnote}} and then we can add more text with another footnote \footnote[frame]{The other footnote}.
\end{frame}

\section{References}
\begin{frame}{Citing and Citations}{Citing}

This is a citation \cite{greenwade93}. You will probably need to compile 
twice to get the reference to show up. 
\end{frame}

\begin{frame}{References}

\bibliographystyle{plain}
\bibliography{example_bib}

\end{frame}

\end{document}